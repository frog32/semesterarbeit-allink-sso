%!TEX root = ../index.tex

\section{Unittests}
\label{sec:unittests}
Die Logik welche die OpenID Konten mit den lokalen Benutzerkonten verbindet wurde erst programmiert, nachdem Unittests dafür programmiert wurden. Diese Unittests laufen auch nachdem das Packet in ein Projekt eingebunden wurde noch und lassen somit ein Test unter Produktivbedingungen zu. Das Protokoll eines Testlaufs ist in Listing~\ref{lst:unittests} zu sehen.

{\minipage{\linewidth}
  \lstinputlisting[
    language=sh,
    breaklines=true,
    caption=Resultat der Unittests,
    basicstyle=\scriptsize,
    captionpos=b,
    label=lst:unittests]{include/unittest.txt}
\endminipage}

\section{Integrationstests}
\label{sec:integrationstests}
Da ein Single-Sign-On System aus mindestens zwei Systemen besteht und in diesem Fall das eine bestehend ist, wird darauf verzichtet automatisierte Integrationstests zu erstellen. Stattdessen werden die wichtigsten Vorgänge mit manuellen Testverfahren getestet.

\subsection{Ausgangslage}
\label{sub:TestingAusgangslage}
Um die Reproduzierbarkeit dieser Tests zu gewährleisten, müssen vor Beginn folgende Schritte getätigt werden.

\begin{itemize}
    \item Das Cache des Browsers muss geleert werden.
    \item Es darf keine offene Google Apps Sitzung bestehen. Falls ein Benutzer
          angemeldet ist, muss dieser abgemeldet werden.
    \item Die Datenbank des Testsystems muss neu erstellt werden.
    \item Sämtliche Cookies welche zum Testsystem gehören müssen gelöscht
          werden.
    \item Ein Testbenutzer ist in Google Apps erstellt und aktiv.
\end{itemize}

\subsection{Testablauf}
\label{sub:testablauf}
\begin{tabular}{p{7cm} p{7cm}}
Aktion & Erwartetes Resultat\\
\hline
Drücken des ``Login mit OpenID'' Buttons im Admin & Der Browser wechselt zum Google Login\\
Eingeben der Benutzerdaten & Django Admin Startseite wird angezeigt und der zugewiesene Benutzer ist angemeldet\\
Löschen der Session des Testsystems und neu laden der Seite & Die Login-Seite erscheint\\
Sperren des Benutzers in Google Apps mit einem anderen Browser & Kein Resultat erwartet\\
Drücken des ``Login mit OpenID'' Buttons im Admin & Der Browser wechselt zum Google Login\\
Eingeben der Benutzerdaten & Benutzer darf sich nicht anmelden können\\
\end{tabular}
