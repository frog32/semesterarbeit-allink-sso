%!TEX root = ../index.tex
\section{Entscheidungen}
\label{sec:Entscheidungen}
\subsubsection{Zu verwendendes System}
\label{ssub:Zu verwendendes System}
Aus der Evaluation geht hervor, dass eine Anbindung an die OpenID Dienste von
Google wohl am meisten Sinn macht. Google-Apps ist ein fester Bestandteil der
Arbeitsmittel jedes Mitarbeiters der Allink.

\subsubsection{Anmelde-Mechanismus}
\label{ssub:Anmelde-Mechanismus}
Da in den meisten Fällen keine granulare Unterteilung der Benutzerrechte jedes
einzelnen Mitarbeiters nötig ist. Da es viel mehr darum geht jedem Mitarbeiter
von Allink Zugriff zu den Admin-Bereichen sämtlicher Webapplikationen zu
gewähren und beim Enden dessen Arbeitsverhältnis diese Rechte wieder zu
entziehen. Wird wie bisher für jede Webapplikation ein Benutzer mit vollen
Administrator-Rechten erstellt. Bei erfolgreichem Anmelden über Google-Apps ist
der Mitarbeiter danach im Admin-Bereich mit diesem Benutzer angemeldet. Der
Mitarbeiter bleibt danach für die Dauer seiner Session angemeldet.

\subsubsection{Wahl eines OpenID Providers}
\label{ssub:Wahl eines OpenID Providers}
Da die Applikation um zu funktionieren nur einen OpenID Provider benötigt. Ist
sie zumindest Theoretisch mit jedem beliebigen OpenID Provider verwendbar. Da
wir die Mitarbeiter mit Hilfe der vom OpenID Provider verlangten E-Mail-Adresse
identifizieren, und ein OpenID Provider diese E-Mail-Adresse fälschen könnte,
kann der OpenID Provider nicht während des Login-Prozesses gewählt werden. Der
Provider muss in den Einstellungen der Webapplikation hinterlegt werden. Falls
jedoch kein Provider hinterlegt wurde, wird das Portal von Google als Standard
verwendet.

\subsection{Abhängigkeiten}
\label{sub:Abhängigkeiten}

\subsubsection{Python OpenID}
\label{ssub:Python OpenID}
Da die korrekte Implementation von OpenID wichtig ist um die Sicherheit zu
gewährleisten, ist es nicht ratsam OpenID ohne eine Library zu verwenden. In
Python existiert zur Zeit nur eine Implementation von OpenID welche auch weiter
entwickelt wird. Diese Library nennt sich python-openid und ist über den
Python-Packet-Index\footnote{http://pypi.python.org/pypi/python-openid/}
installierbar. Weiterentwickelt wird pyton-openid über das zugehörige Github
Repository\footnote{https://github.com/openid/python-openid}.

\subsubsection{Django}
\label{ssub:Django}
Da ein Paket für das Django Webframework erstellt werden soll gehört Django
auch zu den Abhängigkeiten. Um `Class Based Views' benötigt man mindestens
Version 1.3, welche zum Zeitpunkt dieser Arbeit bereits nicht mehr die
aktuellste Version ist.
