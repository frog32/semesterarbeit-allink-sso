%!TEX root = ../index.tex
\section{Entscheidungen}
\label{sec:Entscheidungen}

\subsection{Zu verwendendes System}
\label{sub:Zu verwendendes System}
Aus der Evaluation geht hervor, dass eine Anbindung an die OpenID Dienste von Google wohl am meisten Sinn macht. Google Apps ist ein fester Bestandteil der Arbeitsmittel jedes Mitarbeiters der allink.

\subsection{Anmelde-Mechanismus}
\label{sub:Anmelde-Mechanismus}
In den meisten Fällen ist keine granulare Unterteilung der Benutzerrechte jedes einzelnen Mitarbeiters nötig. Den es geht viel mehr darum, jedem Mitarbeiter von allink Zugriff zu den Administrations-Bereichen sämtlicher Webapplikationen zu gewähren und beim Enden des Arbeitsverhältnisses diese Rechte wieder zu entziehen. Aufgrund dessen wird wie bisher für jede Webapplikation ein Benutzer mit vollen Administrator-Rechten erstellt. Bei erfolgreichem Anmelden über Google Apps ist der Mitarbeiter danach im Administrations-Bereich mit diesem Benutzer angemeldet und bleibt für die Dauer seiner Session zugelassen.
Es ist möglich Regeln für die automatische Zuweisung von lokalen Benutzerkonten zu OpenID Benutzern zu erstellen. Diese Regeln können unterschiedlich gewichtet werden um Fälle in denen zum Beispiel wenige OpenID Benutzer Administrator-Rechte haben und alle übrigen Mitarbeiter Autoren-Rechte erhalten sollen.

\subsection{Wahl eines OpenID-Providers}
\label{sub:Wahl eines OpenID Providers}
Die Applikation benötigt um lauffähig zu sein einen OpenID-Provider. Theoretisch lässt sie sich mit jedem beliebigen OpenID-Provider verwenden. Da wir die Mitarbeiter mit Hilfe der vom OpenID-Provider verlangten E-Mail-Adresse identifizieren und ein OpenID-Provider diese E-Mail-Adresse fälschen könnte, kann der OpenID-Provider nicht während des Login-Prozesses gewählt werden. Der Provider muss in den Einstellungen der Webapplikation hinterlegt werden. Falls jedoch kein Provider hinterlegt wurde, wird das Portal von Google als Standard verwendet.

\subsection{Abhängigkeiten}
\label{sub:Abhängigkeiten}
Da die zu erstellende Applikation in bestehende Projekte integriert werden soll, ist es wichtig die externen Abhängigkeiten so gering wie möglich zu gestalten. So wird nur für die Implementation von OpenID auf eine Bibliothek zurückgegriffen.

\subsubsection{Python OpenID}
\label{ssub:Python OpenID}
Da die korrekte Implementation von OpenID wichtig ist um die Sicherheit zu gewährleisten, ist es nicht ratsam OpenID ohne eine Library zu verwenden. In Python existiert zur Zeit nur eine Implementation von OpenID welche auch weiter entwickelt wird. Diese Library nennt sich python-openid und ist über den Python-Packet-Index\footnote{\url{http://pypi.python.org/pypi/python-openid/}} installierbar. Weiterentwickelt wird pyton-openid über das zugehörige Github Repository\footnote{\url{https://github.com/openid/python-openid}}.

\subsubsection{Django}
\label{ssub:Django}
Da ein Paket für das Django Webframework erstellt werden soll gehört Django auch zu den Abhängigkeiten. Um `Class Based Views' verwenden zu können, benötigt man mindestens Version 1.3, welche zum Zeitpunkt dieser Arbeit bereits nicht mehr die aktuellste Version ist.
