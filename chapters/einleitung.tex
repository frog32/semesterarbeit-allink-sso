%!TEX root = ../index.tex
\section{Ausgangslage}
\label{sec:EinleitungAusgangslage}
Die Agentur allink Betreibt für ihre Kunden diverse Django basierte Webapplikationen. Diese Applikationen besitzen jeweils einen Administrations-Bereich, in welchem man sich mit Benutzername und Passwort einloggen kann. Da allink anfangs 2012 fast 100 solche Webapplikationen in Betrieb hatte, werden vielfach die selben Login-Daten wiederverwendet.

\section{Problemstellung}
\label{sec:Problemstellung}
Nicht alle verwendeten Login-Daten sind sauber dokumentiert. Darum muss man teilweise den Entwickler einer Applikation nach den Login-Daten fragen, oder die gängigsten Passwörter durchprobieren um Zugriff auf den Administrations-Bereich zu erhalten. Zudem gibt es keine Möglichkeit einem Mitarbeiter welcher die Agentur verlässt den Zugang zu sperren, ohne dass in sämtlichen Applikationen die Zugangsdaten geändert werden müssen.

\section{Zielsetzung}
\label{sec:Zielsetzung}
Das Hauptziel dieser Arbeit besteht darin, den Zugang zu den Administrations-Bereichen über ein zentrales System zu Regeln. Dadurch kann sich jeder Mitarbeiter mit seinen eigenen Zugangsdaten an allen Administrations-Bereichen anmelden.
