%!TEX root = ../index.tex
\section{Opensource}
\label{sec:opensource}
Da es der Philosophie von allink gehört, entwickelte Tools welche auch für andere Agenturen und Webentwickler von Nutzen sein könnten zu veröffentlichen. Dies aus zwei verschiedenen Gründen auf welche hier kurz eingegangen wird.

\paragraph{Steigerung der Reputation}
\label{par:steigerung_der_reputation}
Um allink für potentielle Mitarbeiter im Webbereich attraktiver zu machen.
\paragraph{Chance auf Mitarbeit Dritter}
\label{par:chance_auf_mitarbeit_dritter}
Wenn ein von allink veröffentlichtes Tool von Dritten genutzt wird, besteht die Chance, dass diese das Tool weiterentwickeln oder bestehende Fehler beheben.

\subsection{Lizenz}
\label{sub:lizenz}
Um ein Tool für Dritte nutzbar zu machen und um zu gewährleisten, dass diese auch den Quellcode des Tools ändern dürfen, muss das ganze Tool unter einer Softwarelizenz, welche dies gestattet, veröffentlicht werden. Die drei Klausel BSD Lizenz\footnote{\url{http://www.opensource.org/licenses/BSD-3-Clause}} lässt dem Lizenznehmer genügend Freiheiten um den Quellcode zu verwenden wo er will. Das Framework Django ist ebenfalls unter der drei Klausel BSD Lizenz veröffentlich worden.

\subsection{Versionsverwaltung}
\label{sub:versionsverwaltung}
Es wurde \glossary{Git} für die Versionierung verwendet da dies dem Standard der allink gehört. Es wurde ein zentrales Repository\footnote{\url{https://github.com/frog32/django-admin-sso}} auf Github eingerichtet um den Quellcode zu veröffentlichen.

\section{Python Paket}
\label{sec:Python Paket}
Es gibt zwei grundsätzlich verschiedene Distributionsarten für Python Pakete. ``Source'' Distributionen sind Archive mit den benötigten Python Dateien zusammen mit einem Setup-Script. ``Builded'' Distributionen hingegen sind für ein bestimmtes System gebaut und lassen sich typischerweise über den Systemeigenen Packet Manager oder im Fall von Windows, über ein Installationsprogramm installieren. Python Pakete welche für den Gebrauch in Webapplikationen konzipiert sind, sind normalerweise als ``Source''-Distribution erhältlich. Diese ``Source''-Distributionen werden dann mit easy\_install\footnote{\url{http://packages.python.org/distribute/easy_install.html}} oder pip\footnote{\url{http://www.pip-installer.org/}} Installiert. Alternativ können ``Source''-Distributionen auch durch simples Entpacken und Ausführen des Installations-Skripts installiert werden. Da ``django-admin-sso'' nur für die Verwendung in Webapplikationen vorgesehen ist, wird nur eine ``Source''-Distribution erstellt.

\subsection{Paketerstellung}
\label{sub:paketerstellung}
Ein Python-Paket wird durch verschiedene Dateien charakterisiert. Auf die für die Distribution verwendete Tateien wird in Tabelle~\ref{tab:paket_inhalt} eingegangen. Im Manifest wurden zusätzlich noch nicht existierende Dateien erwähnt welche für eine Übersetzung des Paketes in mehrere Sprachen verwendet werden.

\begin{table}[h]
  \centering
  \begin{tabular}{|l|l|p{7cm}|}
  \hline
  Dateiname & Zwingend & Inhalt\\
  \hline
  AUTHORS & Nein & eine Liste aller Authoren\\
  \hline
  LICENSE & Nein & die verwendete Softwarelizenz\\
  \hline
  MANIFEST.in & Nein & eine Liste von nicht Python Dateien welche berücksichtigt werden muss\\
  \hline
  setup.cfg & Nein & Einstellungen für Dokumentationstools\\
  \hline
  setup.py & Ja & Konfiguration des Installationsprogramms\\
  \hline
  \end{tabular}
  \label{tab:paket_inhalt}
  \caption{Dateien und ihre Funktion für eine Distribution}
\end{table}

\subsection{Registration}
\label{sub:registration}
Der ``Python Packet Index''\footnote{\url{http://pypi.python.org/}} bildet die zentrale Registrierungsstelle für alle Python Pakete. Es ist ratsam jedes Paket welches für den freien Gebrauch bestimmt ist in diesem Index zu registrieren.\footnote{\url{http://pypi.python.org/pypi/django-admin-sso/}}

Zusätzlich zum ``Python Packet Index'' wurde das Paket auch in den Index von ``Django Packages''\footnote{\url{http://www.djangopackages.com/packages/p/django-admin-sso/}} eingetragen. Diese Platform wird von vielen Django-Entwicklern verwendet um neue Pakete zu evaluieren.
