%!TEX root = ../index.tex
Zu Beginn der Arbeit standen zwei Termine fest. Vom 16.4.2012 bis am 4.5.2012 musste ich die Arbeit unterbrechen, da ich in dieser Zeit meinen letzten ``Fortbildungsdienst der Truppe''\footnote{\url{http://www.vtg.admin.ch/internet/vtg/de/home/militaerdienst/dienstleistende/dienstleistungspflicht/sdt.html}} bei der Schweizer Armee leistete. Die Arbeit sollte spätestens in der Woche vom 16.7.2012 abgeben werden, da ich danach drei Wochen abwesend sein werde.

\section{Projektplan}
\label{sec:projektplan}
Das Projekt wurde nur auf Wochen genau geplant, da zu Beginn der Arbeit nicht klar war, wann ich wie viel Zeit in die Arbeit investieren kann. Diese Planung ist in Abbildung~\ref{fig:zeitplan} zu sehen. Die gesetzten Meilensteine wurden nicht alle termingerecht erreicht. Jedoch war die Implementierung mit einer Woche Verspätung fertig und auch bereits in der allink im Einsatz.

\section{Zeitaufwände}
\label{sec:zeitaufwände}
Die Tabelle~\ref{tab:zeitabrechnung} zeigt die für die Semesterarbeit aufgewendete Zeit. Nicht in dieser Tabelle enthalten sind diverse kleinere Aufwände, Sitzungen und Schulungen in der allink und der Lightningtalk an der Djangoconeu 2012.
\begin{table}[ht]
  \centering
  \begin{tabular}{lr}
    Planung & 14h \\
    Evaluation & 10h \\
    Design & 12h \\
    Testing & 4h \\
    Implementation & 31h \\
    Dokumentation & 50h \\
    \hline
    Total & 121h \\
  \end{tabular}
  \caption{Zeitabrechnung}
  \label{tab:zeitabrechnung}
\end{table}

\begin{figure}
  \centering
	\includegraphics[width=21cm, angle=90]{include/zeitplan.png}
	\caption{Zeitplan}
	\label{fig:zeitplan}
\end{figure}
