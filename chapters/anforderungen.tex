%!TEX root = ../index.tex
Die Anforderungen an eine Single-sign-On Lösung wurden bereits lange vor dieser Arbeit durch die technische Leitung der allink festgelegt. In diesen Anforderungen wird mit Benutzer ein Mitarbeiter der allink gemeint und mit Administrator ein Mitglied der technischen Leitung

\section{Funktionale anforderungen}
\label{sec:funktionale_anforderungen}
Die Single-sign-On Lösung sollte mindestens folgende Funktionalitäten erfüllen:
\begin{enumerate}
  \item Ein Benutzer kann sich mit seinen globalen Login-Daten an sämtlichen Webapplikationen der allink anmelden für die er berechtigt ist.
  \item Grundsätzlich sind alle Benutzer berechtigt sich an allen Webapplikationen anzumelden. Dies kann für einzelne Applikationen weiter eingeschränkt werden.
  \item Lokale Benutzerkonten welche mit einem eigenen Passwort funktionieren werden nur noch für die Kunden von allink verwendet.
  \item Ein neuer Benutzer kann an einem zentralen System erfasst werden, er hat sofort die Rechte für alle Webapplikationen welche keine besonderen Beschränkungen haben.
  \item Ein Benutzer kann zentral gesperrt werden. Danach ist es ihm nicht mehr möglich sich anzumelden. Bestehende Sitzungen werden nicht abgebrochen.
\end{enumerate}

\section{Technische Anforderungen}
\label{sec:technische_anforderungen}
Da diese SSO-Lösung für sehr viele Projekte eingesetzt werden soll ist es wichtig, dass der nötige Aufwand um dies in ein Projekt zu integrieren möglichst klein gehalten wird.