%!TEX root = ../index.tex
Das Szenario, dass sich ehemalige Mitarbeiter an Passwörter erinnern und mit diesem Wissen versuchen der Agentur oder einem Kunden Schaden zuzufügen, hat sich bisher glücklicherweise nicht ergeben. Jedoch können wir nun mit einfachen Mitteln ehemaligen Mitarbeitern den Zugang zu  von uns betreuten Webseiten sperren. Auch wenn in der Praxis wahrscheinlich nie Probleme entstehen würden, lässt uns der Einsatz von django-admin-sso besser schlafen.

Diese Arbeit hat ihren Ursprung im Herbst 2011, als wir uns in der allink ein Zeitbudget von zwei Tagen reserviert hatten um alle Mitarbeiter über ein zentrales System zu autorisieren. Schnell hatte sich damals gezeigt, dass in den zwei Tagen keine praktikable Lösung erzielt werden konnte. Daraufhin entstand die Idee meine Semesterarbeit dem Thema zu widmen.

Ich konnte durch diese Arbeit an der Djangoconeu 2012 viele Kontakte zu Webentwicklern knüpfen, welche das gleiche Problem hatten wie wir. Von diesen setzen heute zumindest einige meine Applikation ein. Ich konnte zudem im Zuge der Arbeit meine Kenntnisse in diversen weniger oft verwendeten Gebieten auffrischen oder vertiefen. Zu diesen Gebieten gehören unter anderem das erstellen eines Python Paketes, erstellen von Dokumentationen mit LaTeX und das Anwenden von Projektplanungstools.