%  --------------------------------------------------------------------------
%  Semesterarbeit Single Sign-On Lösung für Django Dokumentation
%  Created by Marc Egli on 2012-04-02.
%  --------------------------------------------------------------------------

%  --------------------------------------------------------------------------
%  Latex Document Settings
%  --------------------------------------------------------------------------
\documentclass[
11pt, % Schriftgrösse
a4paper, % A4 Papier
BCOR10mm, % Absoluter Wert der Bindekorrektur, z.B. BCOR1cm
DIV14, % Satzspiegel festlegen siehe
       % http://www.ctex.org/documents/packages/nonstd/koma-script.pdf
footsepline = false, % Trennlinie zwischen Textkörper und Fußzeile
                     % bei normalen Seiten
headsepline, % Trennlinie zwischen Kopfzeile und Textkörper
             % bei normalen Seiten
oneside, % Zweiseitig
openright,
halfparskip, % Europäischer Satz mit Abstand zwischen den Absätzen
abstracton, % inkl. Abstract
listof=totocnumbered, % Abb.- und Tab.verzeichnis im Inhaltsverzeichnis
bibliography=totocnumbered % Lit.zeichnis in Inhaltsverzeichnis aufnehmen
]{scrreprt}

\usepackage[automark]{scrpage2} % Gestaltung von kopf- und Fußzeile
\usepackage[ngerman]{babel}
\usepackage[ngerman]{translator}
\usepackage{tocbasic}
\usepackage[utf8]{inputenc}
\usepackage{lmodern} % Latin Modern
\usepackage[T1]{fontenc}
\usepackage{hyphenat}
\usepackage{ae} % Schöne Schriften für PDF-Dateien

% Tradmark
\def\TTra{\textsuperscript{\texttrademark}}

%1.5 Zeilenabstand
\usepackage[onehalfspacing]{setspace}

% Festlegung des Seitenstils (scrpage2)
\pagestyle{scrheadings}
\clearscrheadfoot
\automark[chapter]{section}

% \lehead{\sffamily\upshape\headmark}
% \cehead{}
% \rehead{}
% \lefoot[\pagemark]{\upshape \pagemark}
% \cefoot{}
% \refoot{}
% \lohead{}
% \cohead{}
\lohead{\sffamily\upshape\headmark}
\lofoot{}
\cofoot{}
\rofoot[\pagemark]{\scshape \pagemark}

% Surround parts of graphics with box
\usepackage{boxedminipage}

% Package for including code in the document
\usepackage{listings}

% If you want to generate a toc for each chapter (use with book)
\usepackage{minitoc}
\usepackage{longtable}

% Abkürzungsverzeichnis erstellen.
\usepackage[printonlyused]{acronym}

% schöne Tabelle zeichnen
\usepackage{booktabs}
\renewcommand{\arraystretch}{1.4} %Die Zeilenabstände in Tabllen angepasst.

% für variable Breiten
\usepackage{tabularx}

% Durchgestrichener Text
\usepackage[normalem]{ulem} %emphasize weiterhin kursiv

% This is now the recommended way for checking for PDFLaTeX:
\usepackage{ifpdf}

\usepackage{eurosym}

\usepackage{natbib}

\usepackage{paralist}

\usepackage{array,ragged2e}

\usepackage[hyperfootnotes=false]{hyperref}
\hypersetup{
  bookmarks=true,         % show bookmarks bar?
  unicode=true,           % non-Latin characters in Acrobat’s bookmarks
  pdftoolbar=true,        % show Acrobat’s toolbar?
  pdfmenubar=true,        % show Acrobat’s menu?
  pdffitwindow=true,      % window fit to page when opened
  pdfstartview={FitH},    % fits the width of the page to the window
  pdftitle={Semesterarbeit},   
  pdfauthor={Marc Egli},
  pdfsubject={Single Sign-On Lösung für Django},
  pdfcreator={TeX Live 2011},
  pdfproducer={pdfTeX, Version 3.1415926-2.3-1.40.12},
  pdfnewwindow=true,      % links in new window
  colorlinks=true,       % false: boxed links; true: colored links
  % linkcolor=blue,          % color of internal links
  % citecolor=black,        % color of links to bibliography
  % filecolor=magenta,      % color of file links
  % urlcolor=cyan          % color of external links
  linkcolor=black,          % color of internal links
  citecolor=black,        % color of links to bibliography
  filecolor=black,      % color of file links
  urlcolor=black          % color of external links
}

\ifpdf
    \usepackage[pdftex]{graphicx}
\else
    \usepackage{graphicx}
\fi

\makeatletter 
\let\orgdescriptionlabel\descriptionlabel 
\renewcommand*{\descriptionlabel}[1]{% 
  \let\orglabel\label 
  \let\label\@gobble 
  \phantomsection 
  \edef\@currentlabel{#1}% 
  %\edef\@currentlabelname{#1}% 
  \let\label\orglabel 
  \orgdescriptionlabel{#1}% 
} 
\makeatother 

%  --------------------------------------------------------------------------
%  Start Document
%  --------------------------------------------------------------------------
\title{Single Sign-On Lösung für Django}

\author{Semesterarbeit in Informatik\\
    \\
    Studierender - Marc Egli\\
	Auftraggeber - Silvan Spross\\
    Projektbetreuer - Lukas Eppler\\
	\\
	HSZ-T - Technische Hochschule Zürich}

\date{Februar 2012 bis ...}


\begin{document}
  \ifpdf
    \DeclareGraphicsExtensions{.pdf, .jpg, .tif}
  \else
    \DeclareGraphicsExtensions{.eps, .jpg}
  \fi

  \pagenumbering{Alph}
  
  \maketitle
  %!TEX root = ../index.tex
Diese Semesterarbeit widmet sich einem den meisten Webagenturen welche für ihre Kunden viele Webseiten betreuen, bekannten Problem: Zugangsdaten werden oftmals zentral und allen Mitarbeitern welche Zugang zu den Webseiten benötigen zugänglich abgelegt um den schnellen Zugriff auf Administrationsbereiche zu gewährleisten. Dies geschieht zum Wohle des Kunden und im Vertrauen darauf, dass alle Mitarbeiter mit guter Absicht handeln und auch nach dem Verlassen der Agentur nicht die Zugangsdaten die ihnen anvertraut wurden missbrauchen.

Als Ergebnis dieser Arbeit entstand ``django-admin-sso''. Es handelt sich dabei um eine Django Applikation welche Mitarbeiter über ein zentrales System autorisiert und dadurch eine Lösung für das zuvor erwähnte Problem bietet.


  \pagenumbering{Roman}
  
  \tableofcontents
  
  \pagenumbering{arabic}
  
  \chapter{Einleitung}
  \label{cha:Einleitung}
  %!TEX root = ../index.tex
\section{Ausgangslage}
\label{sec:EinleitungAusgangslage}
Die Agentur allink.creative Betreibt für ihre Kunden diverse Django basierte 
Webapplikationen. Diese Applikationen besitzen jeweils einen 
Administrations-Bereich, in welchem man sich mit Benutzername und Passwort 
einloggen kann. Da allink.creative anfangs 2012 fast 100 solche 
Webapplikationen in Betrieb hat, werden die Login-Daten zum Teil mehrfach 
verwendet.

\section{Problemstellung}
\label{sec:Problemstellung}
Nicht alle verwendeten Login-Daten sind sauber dokumentiert. Darum muss man 
teilweise den Entwickler einer Applikation nach den Login-Daten fragen, oder 
die gängigsten Passwörter durchprobieren um Zugriff auf den 
Administrations-Bereich zu erhalten. Zudem gibt es keine Möglichkeit einem 
Mitarbeiter welcher die Agentur verlässt den Zugang zu Sperren, ohne dass man 
in sämtlichen Applikationen die Zugangsdaten ändert.

\section{Zielsetzung}
\label{sec:Zielsetzung}
Das Hauptziel dieser Arbeit besteht darin, den Zugang zu den Administrations-Bereichen über ein zentrales System zu Regeln.


  \chapter{Design}
  \label{cha:Design}
  %!TEX root = ../index.tex
\section{Entscheidungen}
\label{sec:Entscheidungen}

\subsection{Zu verwendendes System}
\label{sub:Zu verwendendes System}
Aus der Evaluation geht hervor, dass eine Anbindung an die OpenID Dienste von Google wohl am meisten Sinn macht. Google Apps ist ein fester Bestandteil der Arbeitsmittel jedes Mitarbeiters der Allink.

\subsection{Anmelde-Mechanismus}
\label{sub:Anmelde-Mechanismus}
Da in den meisten Fällen keine granulare Unterteilung der Benutzerrechte jedes einzelnen Mitarbeiters nötig ist. Da es viel mehr darum geht jedem Mitarbeiter von Allink Zugriff zu den Admin-Bereichen sämtlicher Webapplikationen zu gewähren und beim Enden dessen Arbeitsverhältnis diese Rechte wieder zu entziehen. Wird wie bisher für jede Webapplikation ein Benutzer mit vollen Administrator-Rechten erstellt. Bei erfolgreichem Anmelden über Google Apps ist der Mitarbeiter danach im Admin-Bereich mit diesem Benutzer angemeldet. Der Mitarbeiter bleibt danach für die Dauer seiner Session angemeldet.

\subsection{Wahl eines OpenID Providers}
\label{sub:Wahl eines OpenID Providers}
Da die Applikation um zu funktionieren nur einen OpenID Provider benötigt. Ist sie zumindest Theoretisch mit jedem beliebigen OpenID Provider verwendbar. Da wir die Mitarbeiter mit Hilfe der vom OpenID Provider verlangten E-Mail-Adresse identifizieren, und ein OpenID Provider diese E-Mail-Adresse fälschen könnte, kann der OpenID Provider nicht während des Login-Prozesses gewählt werden. Der Provider muss in den Einstellungen der Webapplikation hinterlegt werden. Falls jedoch kein Provider hinterlegt wurde, wird das Portal von Google als Standard verwendet.

\subsection{Abhängigkeiten}
\label{sub:Abhängigkeiten}
Da die zu erstellende Applikation in bestehende Projekte integriert werden soll, ist es wichtig die externen Abhängigkeiten so gering wie möglich zu gestalten. So wird nur für die Implementation von OpenID auf eine Bibliothek zurückgegriffen.

\subsubsection{Python OpenID}
\label{ssub:Python OpenID}
Da die korrekte Implementation von OpenID wichtig ist um die Sicherheit zu gewährleisten, ist es nicht ratsam OpenID ohne eine Library zu verwenden. In Python existiert zur Zeit nur eine Implementation von OpenID welche auch weiter entwickelt wird. Diese Library nennt sich python-openid und ist über den Python-Packet-Index\footnote{\url{http://pypi.python.org/pypi/python-openid/}} installierbar. Weiterentwickelt wird pyton-openid über das zugehörige Github Repository\footnote{\url{https://github.com/openid/python-openid}}.

\subsubsection{Django}
\label{ssub:Django}
Da ein Paket für das Django Webframework erstellt werden soll gehört Django auch zu den Abhängigkeiten. Um `Class Based Views' verwenden zu können benötigt man mindestens Version 1.3, welche zum Zeitpunkt dieser Arbeit bereits nicht mehr die aktuellste Version ist.

  
  
  \appendix
  \listoffigures
  \listoftables
  % \lstlistoflistings
  
  \bibliographystyle{unsrtnat}
  \bibliography{literaturverzeichnis}
\end{document}
