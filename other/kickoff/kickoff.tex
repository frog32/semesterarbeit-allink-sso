%
%  Kickoff Protokoll
%
%  Created by Marc Egli on 2012-02-07.
%  Template by Silvan Spross
%
\documentclass[]{scrreprt}
\usepackage[ngerman]{babel}

% Use utf-8 encoding for foreign characters
\usepackage[utf8]{inputenc}

% Setup for fullpage use
\usepackage{fullpage}

% Running Headers and footers
%\usepackage{fancyhdr}

% Multipart figures
%\usepackage{subfigure}

% More symbols
%\usepackage{amsmath}
%\usepackage{amssymb}
%\usepackage{latexsym}

% Surround parts of graphics with box
\usepackage{boxedminipage}

% Package for including code in the document
\usepackage{listings}

% If you want to generate a toc for each chapter (use with book)
\usepackage{minitoc}

% This is now the recommended way for checking for PDFLaTeX:
\usepackage{ifpdf}
\usepackage{url}

\ifpdf
    \usepackage[pdftex]{graphicx}
\else
    \usepackage{graphicx}
\fi

\title{Kick-off Protokoll}
    
\author{Studierender - Marc Egli\\
    Projektbetreuer - Lukas Eppler\\
    Auftraggeber - Silvan Spross, allink.creative\\
    \\
    HSZ-T - Technische Hochschule Zürich}
    
\date{1. Februar 2012}

\begin{document}

    \ifpdf
        \DeclareGraphicsExtensions{.pdf, .jpg, .tif}
    \else
        \DeclareGraphicsExtensions{.eps, .jpg}
    \fi

    \maketitle

    \pagenumbering{arabic}

    % \tableofcontents

    \chapter{Kick-off Protokoll}

    \section{Semesterarbeit}
    Single-Sign-On - Lösung für allink GmbH

    \section{Fragen Kick-Off-Meeting}
    Alle Fragen aus dem Kick-Off-Meeting konnten geklärt werden:
    \begin{enumerate}
        \item Steht das auftraggebende Unternehmen hinter dieser Semesterarbeit? \\
            {\bf Ja}, das Unternehmen steht voll hinter dieser Semesterarbeit.
        \item Sind die fachliche Kompetenz und die Verfügbarkeit der Betreuungsperson sichergestellt? \\
            {\bf Ja}, da die Betreuungsperson selbst im Webbereich tätig ist und dieses Fach zusätzlich selbst unterrichtet.
        \item Sind die Urheberrechte und Publikationsrechte geklärt? \\
            {\bf Ja}, die Arbeit kann vollständig öffentlich zugänglich gemacht
            werden und wird unter einer Opensource-Lizenz veröffentlicht.
        \item Bekommt die Studentin oder der Student die notwendige logistische und beratende
            Unterstützung durch das auftraggebende Unternehmen? \\
            {\bf Ja}
        \item Entsprechen Thema und Aufgabenstellungen den Anforderungen an eine Semesterarbeit? \\
            {\bf Ja}
        \item Ist die Arbeit klar abgegrenzt und terminlich entkoppelt von den Prozessen des auftraggebenden Unternehmens? \\
            {\bf Ja}
        \item Ist eine Grobplanung vorhanden? Sind die nächsten Schritte \\
            klar formuliert (von der Studentin oder dem Studenten)? \\
            {\bf Ja}
        \item Ist die Arbeit technisch und terminlich von der Studentin oder dem Studenten umsetzbar? \\
            {\bf Ja}
    \end{enumerate}
    
    \section{Beschlüsse}
    Folgende weitere Beschlüsse wurden während des Meetings gefällt.
    \begin{itemize}
        \item Als Projektplanungstool soll Basecamp verwendet werden.
        \item Sowohl die Dokumentation wie auch der Programmcode soll auf Github gespeichert werden.
        \item Da bei einer Single-Sign-On Schnittstelle wenig Änderungen zu erwarten sind, soll ein ``Big Design Up Front'' Ansatz verwendet werden.
        \item Während der Programmierphase soll ``Test driven Development'' angewendet werden.
    \end{itemize}
    
    \section{Weiteres}
    Weiterhin wurden folgende Themen besprochen:
    \begin{itemize}
        \item Das Buch ``The Pyramid Principle'' ist eine gute Hilfe um Dokumentationen richtig aufzubauen.
        \item Kontaktdaten wurden ausgetauscht.
    \end{itemize}
    
    \section{Geplante Milestones}
    \begin{tabular}{l r}
        Planung abgeschlossen & 17.2.2012 \\
        Evaluation und Design abgeschlossen & 23.3.2012 \\
        Testszenario erstellt & 6.4.2012 \\
        Implementation fertig & 18.5.2012 \\
    \end{tabular}
    
\end{document}