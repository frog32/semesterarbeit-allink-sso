%
%  Design Review Protokoll
%
%  Created by Marc Egli on 2012-05-15.
%  Template by Silvan Spross
%
\documentclass[]{scrreprt}
\usepackage[ngerman]{babel}

% Use utf-8 encoding for foreign characters
\usepackage[utf8]{inputenc}

% Setup for fullpage use
\usepackage{fullpage}

% Running Headers and footers
%\usepackage{fancyhdr}

% Multipart figures
%\usepackage{subfigure}

% More symbols
%\usepackage{amsmath}
%\usepackage{amssymb}
%\usepackage{latexsym}

% Surround parts of graphics with box
\usepackage{boxedminipage}

% Package for including code in the document
\usepackage{listings}

% If you want to generate a toc for each chapter (use with book)
\usepackage{minitoc}

% This is now the recommended way for checking for PDFLaTeX:
\usepackage{ifpdf}
\usepackage{url}

\ifpdf
    \usepackage[pdftex]{graphicx}
\else
    \usepackage{graphicx}
\fi

\title{Design-Review Protokoll}
    
\author{Studierender - Marc Egli\\
    Projektbetreuer - Lukas Eppler\\
    Auftraggeber - Silvan Spross, allink.creative\\
    \\
    HSZ-T - Technische Hochschule Zürich}
    
\date{5. Mai 2012}

\begin{document}

    \ifpdf
        \DeclareGraphicsExtensions{.pdf, .jpg, .tif}
    \else
        \DeclareGraphicsExtensions{.eps, .jpg}
    \fi

    \maketitle

    \pagenumbering{arabic}

    % \tableofcontents

    \chapter{Design-Review Protokoll}

    \section{Semesterarbeit}
    Single-Sign-On - Lösung für allink GmbH

    \section{Beschlüsse}
    \label{sec:Beschlüsse}
    Folgende Beschlüsse wurden während dem Meeting gefällt.
    \begin{itemize}
        \item Das weitere Vorgehen wurde festgehalten.
        \item Die Möglichkeit LDAP für Webapplikationen zu verwenden soll abgeklärt werden.
        \item Es soll anhand eines Regelsets für jeden neuen Benutzer ein lokaler Benutzer gewählt werden oder der Zugriff verweigert werden.
        \item Die Semesterarbeit ist mit der bestehenden Aufgabenstellung umsetzbar.
    \end{itemize}
    
    \section{Weiteres}
    \label{sec:Weiteres}    
    Weiterhin wurden folgende Themen besprochen:
    \begin{itemize}
        \item Milestones welche abgeschlossen sind sollten im Basecamp auch geschlossen werden.
        \item Der Auftraggeber ist erfreut über den Fortschritt der Arbeit
    \end{itemize}
    
\end{document}
